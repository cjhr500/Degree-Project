\documentclass[authoryearcitations]{UoYCSproject}
\author{Caleb J. H. Riley}
\title{Evolutionary agent-based simulation modelling of human life-history evolution}
\date{Version 0.01, 2016-November-15}
\supervisor{Daniel W. Franks}
\BEng

\wordcount{2001}

\includes{}

\excludes{}

\abstract{This is an abstract. Should be about 500 words long.}

\dedication{}

\acknowledgements{}

\usepackage{framed}
\usepackage{hyperref}

\begin{document}
\maketitle
\listoffigures
\listoftables

\cleardoublepage

\chapter{Introduction}
\label{cha:Introduction}
This should be about 1000 words long.

\chapter{Literature Review}
\label{cha:Literature Review}
This should be about 3000 words long.

\section{What is Menopause?}

\section{Modelling techniques}
Deterministic vs stochastic -- computers provide new methods
\subsection{Deterministic Models}
\subsection{Stochastic Models}

\section{Theories to explain evolution of menopause}

\begin{framed}
\noindent \textbf{The evolution of prolonged life after reproduction. \cite{evolutionPRLS2015}}

Overview paper reviewing previous research into the presence of and theories for the existence of long post reproductive lifespans (PRLS).

Found in Humans, Killer Whales/Orcas and Short finned pilot whales.

\noindent Non-adaptive hypotheses:
\begin{itemize}
    \item extended lifespans caused by improvements in medicine
    \item males preferring younger females
\end{itemize}

\noindent Adaptive hypotheses:
\begin{itemize}
    \item mother hypothesis - to look after previous offspring rather than having new ones.
    \item grandmother hypothesis - to look after grandchildren to enable daughter(in law) to have more children.
    \item reproductive conflict hypothesis - grandmother’s children competing with children
\end{itemize}
\end{framed}

\subsection{Mother Hypothesis}
\subsection{Grandmother Hypothesis}
\subsection{Male Preference}
\begin{framed}
\noindent \textbf{Why Men Matter: Mating Patterns Drive Evolution of Human Lifespan \cite{whyMenMatter2007}}

There is a lack of a wall of death - females dying immediately after menopause - when using a two-sex model opposed to a one-sex model.

Older males prefer younger females in the model as females their own age may be post-reproductive.

This preference reinforces post-reproductive lifespans as females are not reproducing due to the lack of male interest - thus the biological need for them to remain reproductive is diminished.

\noindent \textbf{My Thoughts}

There seems to be no accounting for the fact that male preference for younger females could have developed after the evolution of long post-reproductive females.

Indeed it seems that the evolution of a longer period of female reproduction would occur as those who remained fertile for longer would likely be still reproduced with, producing offspring with genes that can reproduce for longer. This extended period of reproduction would also probably result in more offspring than those who stopped reproducing at a younger age.

The statistical model is poorly explained - it is unclear how male preference has been implemented.
\end{framed}

\begin{framed}
\noindent \textbf{Patriarch hypothesis. \cite{patriarchHypothesis2000}}

Notes

\noindent \textbf{My Thoughts}


\end{framed}


\begin{framed}
\noindent \textbf{Mate Choice and the Origin of Menopause. \cite{mateChoice2013}}

Notes

\noindent \textbf{My Thoughts}


\end{framed}


\subsection{Reproductive Conflict}
\begin{framed}
\noindent \textbf{Example 1.}

Admittedly, this is a very simplistic description of what really happens, but the point is that TeX operates with glue and boxes. Letters are not the only things that can be boxes. One can put virtually everything into a box, including other boxes. Each box will then be handled by LaTeX as if it were a single letter.
\end{framed}

\section{Male preference and modelling}

\chapter{Problem Description/Analysis}
\label{cha:Problem Description}
This should be about 1500 words long.

\chapter{Design and Implemenation}
\label{cha:Design and Implementation}
This should be about 2500 words long.

\chapter{Results and Evaluation}
\label{cha:Results and Evaluation}
This should be about 2500 words long.

\chapter{Conclusion}
\label{cha:Conclusion}
This should be about 1000 words long.

\bibliographystyle{plain}
\bibliography{project} 
\end{document}
