\documentclass[authoryearcitations]{UoYCSproject}
\author{Caleb J. H. Riley}
\title{Evolutionary agent-based simulation modelling of human life-history evolution}
\date{Version 0.01, 2016-November-15}
\supervisor{Daniel W. Franks}
\BEng

\wordcount{8832}

\includes{}

\excludes{}

\abstract{This is an abstract. Should be about 500 words long.}

\dedication{}

\acknowledgements{}


\usepackage{framed}
\usepackage{hyperref}

\begin{document}
\maketitle
\listoffigures
\listoftables

\cleardoublepage

\chapter{Introduction}
\label{cha:Introduction}
This should be about 1000 words long.

\chapter{Literature Review}
\label{cha:Literature Review}
This should be about 3000 words long.

\section{What is Menopause?}

\section{Modelling techniques}
Deterministic vs stochastic -- computers provide new methods
\subsection{Deterministic Models}
\subsection{Stochastic Models}

\section{Theories to explain evolution of menopause}
\subsection{Mother Hypothesis}
\subsection{Grandmother Hypothesis}
\subsection{Male Preference}
\subsection{Reproductive Conflict}
\begin{framed}
\noindent \textbf{Example 1.}

Admittedly, this is a very simplistic description of what really happens, but the point is that TeX operates with glue and boxes. Letters are not the only things that can be boxes. \cite{whymenmatter2007} One can put virtually everything into a box, including other boxes. Each box will then be handled by LaTeX as if it were a single letter.
\end{framed}

\section{Male preference and modelling}

\chapter{Problem Description/Analysis}
\label{cha:Problem Description}
This should be about 1500 words long.

\chapter{Design and Implemenation}
\label{cha:Design and Implementation}
This should be about 2500 words long.

\chapter{Results and Evaluation}
\label{cha:Results and Evaluation}
This should be about 2500 words long.

\chapter{Conclusion}
\label{cha:Conclusion}
This should be about 1000 words long.

\bibliographystyle{plain}
\bibliography{project} 
\end{document}
