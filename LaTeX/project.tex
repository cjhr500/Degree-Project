\documentclass[authoryearcitations]{UoYCSproject}
\author{Caleb J. H. Riley}
\title{Evolutionary agent-based simulation modelling of human life-history evolution}
\date{Version 0.01, 2016-November-15}
\supervisor{Daniel W. Franks}
\BEng

\wordcount{2001}

\includes{}

\excludes{}

\abstract{This is an abstract. Should be about 500 words long.}

\dedication{}

\acknowledgements{}

\usepackage{framed}
\usepackage{hyperref}
\usepackage{enumitem}

\begin{document}
\maketitle
\listoffigures
\listoftables

\cleardoublepage

\chapter{Introduction}
\label{cha:Introduction}
\section{Motivation}
Menopause unsolved problem in biology -- many different hypotheses

Better population models

Replicating model can be used to check results found elsewhere

Model can be adapted for use for other hypotheses to do with menopause and/or population modelling.

\section{Aims}
To check validity of existing model by replicating it

To rewrite the model into an object oriented language (python)

Address some of it shortcomings (allowing preference to coevolve)

\section{Thesis Outline}
Chapter Two Literature Review: reviewing past work relevant to the project

Chapter Three Problem Description: looking at what the problem consists of

Chapter Four Design and Implementation: designing a solution to the problem

Chapter Five Results and Evaluation: presenting and analysing the results produced by the solution

Chapter Six Conclusion: making judgement of the solution and the results, suggesting new work.

\section{Statement of Ethics}
Model -- no ethical concerns

\chapter{Literature Review}
\label{cha:Literature Review}

\section{Menopause}
What is menopause. Somatic vs reproductive senescence. 

What animals does it occur in. Wild vs captive.

Possible reasons for menopause Patriarch Hypothesis, Grandmother, reproductive conflict

\subsection{Patriarch hypothesis}
Proposed by \cite{patriarchHypothesis2000} , Males having preference for younger females caused menopause

Deterministic model done in \cite{whyMenMatter2007} but this has fixed age of end of reproduction -- something not true now.

Stochastic model done in \cite{mateChoice2013} -- main focus of report. Fixes many of the flaws of \cite{whyMenMatter2007} (including removing the fixed age of the end of reproduction) but still has problems.


\subsection{Grandmother hypothesis}

\subsection{Reproductive conflict}

\subsection{Other hypotheses}
Follicular depletion, healthcare/lifespan improvements - not evolutionary but epiphenomenon, Risk from late age reproduction.

\section{Evolution}
Overview \cite{origin1859}

\subsection{Key concepts/terms}
\begin{description}[style=nextline]
\item[Selection] Description of selection

\item [Mutation] Description of mutation

\item [Crossover] Description of crossover

\item [Coevolution] Description of co-evolution
\end{description}

\subsection{Biological relevance to project}
Project looking at evolution of long post-reproductive lifespans

\subsection{Computational relevance to project}
Is modelling evolution

\section{Modelling in biology}

\subsection{Deterministic modelling}
Populations often modelled with exponential growth/differential equations

\subsection{Stochastic modelling}
Multiagent systems, genetic algorithms, neural networks, machine learning to reduce dimensionality,

\section{Conclusions from Literature}


\chapter{Problem Description}
\label{cha:Problem Description}
This should be about 1500 words long.

\chapter{Design and Implemenation}
\label{cha:Design and Implementation}
This should be about 2500 words long.

\chapter{Results and Evaluation}
\label{cha:Results and Evaluation}
This should be about 2500 words long.

\chapter{Conclusion}
\label{cha:Conclusion}
This should be about 1000 words long.

\bibliographystyle{plain}
\bibliography{project} 
\end{document}
