\documentclass[authoryearcitations]{UoYCSproject}
\author{Caleb J. H. Riley}
\title{Evolutionary agent-based simulation modelling of human life-history evolution}
\date{Version 0.01, 2016-November-15}
\supervisor{Daniel W. Franks}
\BEng

\wordcount{2001}

\includes{}

\excludes{}

\abstract{This is an abstract. Should be about 500 words long.}

\dedication{}

\acknowledgements{}

\usepackage{framed}
\usepackage{hyperref}
\usepackage{enumitem}

\begin{document}
\maketitle
\listoffigures
\listoftables

\cleardoublepage

\chapter{Introduction}
\label{cha:Introduction}
\section{Motivation}
The senescence of organisms is a long standing evolutionary puzzle: why would genes which cause an organism to degrade over time be selected for? In particular, why could genes which cause a reduction in fertility with age provide evolutionary benefit versus reproducing until death? Menopause is of particular interest since it both only occurs within females, but within limited number of species.

Although we know the the biological cause of menopause, there are many theories about why it is beneficial for a species to cease reproduction long before death. 


\section{Aims}
\subsection{Check validity of computational model}
This project focuses on reproducing a computational model \cite{mateChoice2013} that implements the patriarch hypothesis \cite{patriarchHypothesis2000}, a theory that males' preference for younger females caused long post-reproducitve lifespans. Replicating models is useful to check their validity and discover any flaws they may possess. Replication also provides a jumping off point for improving the model or for adapting it to other hypotheses to do with the evoltion of menopause or population modelling in general. 

\subsection{Rewrite model into an object oriented language}
The model is also being adapted from C, a performant but verbose and old imperative language, into Python, a modern object oriented language designed to be as readable as possible. Python is also popular among scientific research, so this will make the model more accessable.

\section{Thesis Outline}
\begin{description}

\item[Chapter Two Literature Review] The literature review provides an overview of modelling, the different theories for the origin of menopause, and the basic concepts of evolution.
\item[Chapter Three Problem Description] looking at what the problem consists of

\item[Chapter Four Design and Implementation] designing a solution to the problem

\item[Chapter Five Results and Evaluation] presenting and analysing the results produced by the solution

\item[Chapter Six Conclusion] making judgement of the solution and the results, suggesting new work.

\end{description}

\section{Statement of Ethics}
The project is implementing a computational model which is purely theoretical. Although it concerns human reproduction, it does not involve human subjects and implementing it with them would be impossible due to time constraints (as evolution take place over many many years), there are no real ethical concerns.

\chapter{Literature Review}
\label{cha:Literature Review}
Project is covering computational models of populations to try and understand how menopause might be caused by evolution, therefore it is important to talk about them.

\section{Agent Based Modelling}

\subsection{Explantion}

\subsection{Applications}

\section{Genetic Algorithms}

\subsection{Explanation}

\subsection{Applications}

\subsection{Relationship to evolution}

\newpage
\section{Menopause}
Menopause is the process where 

What is menopause. Somatic vs reproductive senescence. Short vs long PRLS.

Which animals has it been show to occur in (Humans, Short Finned Pilot Whales, Orcas (Killer Whales)) Wild vs captive. Human vs Primates.

Possible reasons for menopause Patriarch Hypothesis, Grandmother, reproductive conflict. Overview in \cite{evolutionPRLS2015}

\subsection{Patriarch Hypothesis}
The Patriarch Hypothesis \cite{patriarchHypothesis2000} hypothesises that menopause came about due to older, high status males having access to younger female mates allowing them to reproduce for much longer. This increased the proliferation of geness linked to longevity, increasing both the lifespan of males and females. This is reliant on several factors: 

\begin{itemize}
\item That females have a limited number of oocytes (immature ova) which are depleted over time, and that reproduction naturally comes to an end when they run out. \cite{humanPopBio1994} In early females, before female longevity increased, most females died before their supply of oocytes had been completely depleted, and so did not experience menopause.
\item That longevity causing mutations are on the X rather than the Y chromosome. If the gene were on the Y chromosome then the increase longevity would only be present in the males. The paper suggests that female longevity (and therefore have long post reproductive lifespans) is a result of females being "dragged along" by male longevity being passed on through the X chromosome. 
\item That older men continue to reproduce and pass on their longevity causing genes. High states males (normally those with a better reputation for hunting and gathering) would start a new family with a second, younger wife once their first wife had undergone menopause. Thus males carrying longevity causing genes would have greater opportunity to pass them on.
\end{itemize}

A mathematical model of the hypothesis was created \cite{whyMenMatter2007} which was a set of functions which relied on a group of females aged i, and a group of males aged j, the females' fertility at age i and the mating preference of females of age i for males of age j and vice versa. The main problem with this model is that it had a fixed age at which females stopped reproducing, rather than letting it evolve over time as you would expect in an evolutionary model. Indeed this model can be interpreted to suggest that males' preference for younger females came to be after the evolution of menopause, rather than as a cause of it.

To try and correct some of these shortcomings, a computational model \cite{mateChoice2013} which does not have a fixed age where females become infertile. Each member of the population is modelled as an individual agent, with genes that affect either mortality or fertility, with genes acting either in a sex dependendent or a sex indifferent manner. Pseudo-random numbers are used to determine births, deaths and partnerships against predetermined tables which are modified by genetics. One of the main shortcomings of the model is that it does not allow the preferences of the agents to coevolve with the change in fertility and mortality. 

\subsection{Grandmother Hypothesis and the Mothering Hypothesis}
The grandmother hypothesis \cite{grandmother2000} theorises that the evolution of menopause came about due to the increased inclusive fitness of a population caused by grandmother being able to help her children raise their offspring through providing childcare and resources such as food. This is due to the grandmother being post-reproductive and not having any children that are of an age where they need care. This grandmothering improves the survival of grandchildren and allows her children to reproduce more frequently. Populations which had post-reproductive women would grow more rapidly as a result, meaning they would overtake populations where they did not occur. 

One of the criticisms of the grandmother hypothesis is that the degree of increase to inclusive fitness has been overstated, and that whilst the benefit might be a factor, it is unlikely to be the only factor to have caused it. Another criticism is that if the benefits of grandmothering are so great, why do males remain reproductive throughout their lives? If the increase to inclusive fitness is so great from grandmothering then it seems unlikely that this benefit would be limited ti just one sex.

The mothering hypothesis is a similar theory in which a species improves the survival of offspring by ceasing reproduction at a certain point so that resources could be focused on raising existing children, rather than continuing to focus on reproduction and newborn children. The reduced mortality rate of children would again result in an increase of inclusive fitness.

\subsection{Reproductive Conflict Hypothesis}
Grandmothers stop reproducing so that their offspring are not competing with their grandchildren for resources. \cite{repConflictOrca2017}

\subsection{Other hypotheses}
Follicular depletion, healthcare/lifespan improvements - not evolutionary but epiphenomenon, Risk from late age reproduction.

\newpage
\section{Evolution}
Overview \cite{origin1859}

\subsection{Key concepts/terms}
\begin{description}[style=nextline]
\item[Selection] Selection is the process in which survival and reproduction of a species is affected by the phenotype (physical expression of genetics). A trait that increases lifespan and allows more time for reproduction will cause there to be a greater chance for more offspring with that trait to be produced. By comparison a trait that decreases lifespan or realised fertility will tend to become less prevalent in the population as they have fewer offspring and these offspring if they have the same phenotype will inherit the lower survival chances. In essence useful traits survive, whereas traits that are less useful will gradually disappear.

\item [Mutation] Genetic mutation is the change of genetic information at random, usually from mistakes when copying DNA. 

\item [Crossover] Description of crossover

\item [Coevolution] Description of co-evolution
\end{description}

\section{Conclusions from Literature}


\chapter{Problem Description}
\label{cha:Problem Description}
This should be about 1500 words long.

\chapter{Design and Implemenation}
\label{cha:Design and Implementation}
This should be about 2500 words long.

\chapter{Results and Evaluation}
\label{cha:Results and Evaluation}
This should be about 2500 words long.

\chapter{Conclusion}
\label{cha:Conclusion}
This should be about 1000 words long.

\bibliographystyle{plain}
\bibliography{project} 
\end{document}
